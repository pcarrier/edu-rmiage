\documentclass[a4paper,10pt,twoside]{article}

\usepackage[utf8]{inputenc}
\usepackage[T1]{fontenc}
\usepackage[francais]{babel}

\usepackage{hyperref}

\usepackage{textcomp}

\usepackage{a4wide}
\usepackage{xcolor}
\usepackage{graphicx}
\usepackage{amsmath}
\usepackage{amsfonts}
\usepackage{amssymb}

\usepackage{fancyhdr}
\pagestyle{fancy}

\usepackage[absolute]{textpos}

\begin{document}

\title{Projet RMIAGE\\ \Huge{Plan de développement logiciel} }
\author{
	Badre Baba, Geoffroy Carrier, J-Ch Saad-Dupuy, Mickael Scheer
}

\begin{center}
\begin{textblock*}{35cm}(2mm,3mm)
\includegraphics[scale=0.8]{../cc/imag_logo.png}
\end{textblock*}
\end{center}
 
\maketitle


\section{Introduction}
\subsection{Objectif du document}
Ce document a pour but de présenter l'ensemble des informations nécessaires au
contrôle du projet RMIAGE. Il sert à une définir une bonne gestion du projet.
Il décrit en détail les différentes phases du projet avec leur durée et les ressources nécessaires à leur bon 	déroulement

\subsection{Portée du document}
Ce document sera utilisé par l’équipe du projet et sera disponible pour le client si celui-ci désire l’étudier et faire des remarques. 

\subsection{Définitions, acronymes et abréviations}
Dans ce document, les termes suivants seront utilisés :
\begin{itemize}
	\item Framework :
Il s'agit du produit développé par notre entreprise. Un framework est un plan de travail pour accélérer le travail de développeurs tiers.
	\item Java :
Langage de déveoppement suivant le paradigme de la programmation orientée objet. Il s'agit du langage utilisé pour développer l'application rmiage.
	\item JVM :
Java Virtual Machine, ou machine virtuelle java. Nécéssaire pour éxécuter les applications développées en Java.
	\item RMI :
Remote Method Invocation. Technologie utilisée dans le cadre de notre développement assurant
la gestion client/serveur.
\end{itemize}

\subsection{Références}
\subsection{Vue d’ensemble}

\section{Vue d’ensemble du projet}
\subsection{But du projet, portée et objectifs}
Le projet consiste à réaliser un framework complet et efficace permettant à l'entreprise cliente de fortement alléger ses délais de développement d'applications orientées réseau social.

\subsection{Hypothèses et contraintes}
Le framework devra s'intégrer dans les infrastructures matérielles et logicielles des clients, la réutilisabilité, la généricité et l'interopérabitilité de ce dernier seront indispensables.
Le frameweork devra aussi être développé sur une plateforme Java et sera aussi doté de la technologie RMI assurant la partie communication entre le client et le serveur.

\subsection{Biens livrables du projet}
Le projet sera livré sous forme de trois archives JAR : une pour la partie framework,
ainsi qu'une pour chaque application témoin (client/serveur).

\subsection{Évolution du plan de développement logiciel}
Ce Plan de Développement Logiciel sera certainement amené à être modifié lors du projet, il sera bien évidement à la disposition du client durant toute la durée du projet.

\section{Organisation du projet}
\subsection{Structure d’organisation}
L'équipe est composée de quatre membres.
% TODO lister fonction des membres
\begin{itemize}
	\item Badre Baba ;
	\item Geoffroy Carrier ;
	\item Jean-Christophe Saad-Dupuy ;
	\item Scheer Mickael.
\end{itemize}

\subsection{Interfaces externes}
\subsection{Rôles et Responsabilités}


\section{Processus de gestion}
\subsection{Estimations de projet}

\subsection{Plan de projet}
\subsubsection{Planification des phases}
\subsubsection{Objectifs d’itération}
\subsubsection{Version}
\subsubsection{Calendrier du projet}
\subsubsection{Ressources humaines du projet}

\subsection{Suivi de projet et contrôle}
\subsubsection{Gestion des exigences}
\subsubsection{Contrôle de la qualité}
\subsubsection{Rapports et mesures}
\subsubsection{Gestion de risque}
\subsubsection{Gestion de configuration}



\section{Annexes}

\end{document}
