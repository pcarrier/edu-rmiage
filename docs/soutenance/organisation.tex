\section{Gestion de projet}
\subsection{Equipe de développement}
Au début du projet, nous avons essayer de déterminer des rôles et les tâches qui y sont attachées.

Nous avons déterminés les rôles suivant :
\begin{itemize}
 \item Un chef de projet, dont le rôle principal est d'assurer la coordination de l'équipe~;
 \item Un architecte, chargé de concevoir l'organisation du framework~;
 \item Un dévelopeur, en charge d'implémenter notre solution~;
 \item Un responsable qualité, en charge d'assurer la qualité du produit et sa validité.
\end{itemize}
Le détail de ces reponsabilités est décrit dans la figure \ref{fig:repart_effect} p.\pageref{ }, partie "Rôles modélisés".

Il est apparu que ces rôles n'étaient pas applicable en raison de leur poids différents, d'une part, mais aussi des compétences et aspiration de chacun. Nous avons éstimé la répartition des tâches comme décrit \ref{fig:repart_effect} p.\pageref{fig:repart_effect} partie "Rôles prévue".
La répartition effective est également décrite.

\subsection{Répartition des tâches}
\begin{figure}[thbp]
	\centering
		\includegraphics[angle=90, scale=0.7]{../diagrammes/repartition_taches.pdf}
	\caption{Répartition effective}
	\label{fig:repart_effect}
\end{figure}
\subsubsection{Prévisionnelle}

\subsubsection{Effective}