%!TEX root = cahier_charges_afnor.tex
\section{Expression fonctionnelle du besoin}

\subsection{Fonctions de service et contrainte}
\subsubsection{Fonctions de service principales}
% (qui sont la raison d’être du produit)
%!TEX root = cahier_charges_afnor.tex
Le framework doit permettre de créer et d'utiliser différents type d'objets génériques, permettant aux application l'utilisant de créer :

\begin{itemize}
\item{des utilisateurs}
\item{des groupes d'utilisateurs}
\item{des strucures pour des échanges de fichiers}
\item{des strucures pour des échanges de messages}
\end{itemize}


\subsubsection{Fonctions de service complémentaires}
% (qui améliorent, facilitent ou complètent le service rendu)

Deux programmes de démonstration exploitant les fonctionnalités du framework doivent également être fournis :

\begin{itemize}
 \item un programme serveur basé sur notre produit proposant un système de forums ;
 \item un programme client, exploitant les ressources fournies par le serveur.
%L'interface attendue pour cette application témoin est donnée en \ref{screen_client}
\end{itemize}

De plus, nous fournirons une batterie de tests unitaires garantissant la qualité de chaque fonctionnalité du \emph{framework}.

\subsubsection{Contraintes}
% (limitations à la liberté du concepteur-réalisateur)
\begin{enumerate}
 \item Développement

Le framework sera aussi indépendant que possible, notamment de tout éditeur de serveur SQL grâce à la technologie Hibernate.

 \item Environnement

Notre framework se composera d'une librairie, utilisée par les applications témoins livrées.

L'utilisation de la technologie Java assure une complète indépendance du ou des systèmes d'exploitations hôtes.

Les applications serveur et cliente pourrons être lancée depuis le même poste, en local, ou sur des machines distantes.

Les machines hôtes et serveur devrons chacune disposer d'une Java Virtual Machine (JVM) afin de pouvoir executer ces applications.

\end{enumerate}

\subsection{Critères d’appréciation}
%(en soulignant ceux qui sont déterminants pour l’évaluation des réponses)
%TODO revoir un peu les critères

Les principaux critères d'appréciation seront : 

\begin{itemize}
 \item du coté \emph{framework} :
 \begin{itemize} 
  \item son indépendance ;
  \item sa qualité ;
  \item sa flexibilité ;
 \end{itemize} 
 
 \item du coté applications témoin :
 \begin{itemize}
  \item leur pertinence vis-à-vis des fonctionnalités principales du \emph{framework} ;
  \item la facilité de leur utilisation pour des tests.
 \end{itemize}
\end{itemize}


%\subsection{Niveaux des critères d’appréciation et ce qui les caractérise}
%\subsubsection{Niveaux dont l’obtention est imposée}
%\subsubsection{Niveaux souhaités mais révisables}
