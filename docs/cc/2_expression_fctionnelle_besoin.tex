\section{Expression fonctionnelle du besoin}

\subsection{Fonctions de service et de contrainte}
\subsubsection{Fonctions de service principales}
% (qui sont la raison d’être du produit)
%!TEX root = cahier_charges_afnor.tex
Le framework doit permettre de créer et d'utiliser différents type d'objets génériques, permettant aux application l'utilisant de créer :

\begin{itemize}
\item{des utilisateurs}
\item{des groupes d'utilisateurs}
\item{des strucures pour des échanges de fichiers}
\item{des strucures pour des échanges de messages}
\end{itemize}


\subsubsection{Fonctions de service complémentaires}
% (qui améliorent, facilitent ou complètent le service rendu)
Deux programmes de démonstration exploitant les fonctionnalités du framework serons également
fournis:
\begin{itemize}
 \item un programme serveur, basé sur notre produit ainsi que la technologie RMI, mettant à disposition des ressources ;
 \item un programme client, exploitant les ressources fournies par le serveur. 
\end{itemize}

De plus, nous fournirons une batterie de tests unitaire garantissant la qualité de chaque fonctionnalité du framework.
\subsubsection{Contraintes}
% (limitations à la liberté du concepteur-réalisateur)
\begin{enumerate}
 \item Développement

Le framework doit être au maximum indépendant, notemment au niveau de l'acces au données.
Le client doit avoir la possibilité d'utiliser le systeme de gestion de base de données relationnelle (SGBDR) de son choix.

 \item Environnement
Notre framework se composeras d'une librairie, utilisée les applications témoins livrées.

L'utilisation de la technologie Java assure une compléte indépendance du ou des systèmes d'exploitations hôtes.

Le applications serveur et cliente pourrons être lancée depuis le même poste, en local, ou sur des machines distantes.

Les machines hôtes et serveur devrons éviement chacune disposer d'une Java Virtual Machine (JVM) afin de pouvoir executer ces applications.

\end{enumerate}

\subsection{Critères d’appréciation}
%(en soulignant ceux qui sont déterminants pour l’évaluation des réponses)

\subsection{Niveaux des critères d’appréciation et ce qui les caractérise}
\subsubsection{Niveaux dont l’obtention est imposée}
\subsubsection{Niveaux souhaités mais révisables}
