\section{Expression fonctionnelle du besoin}

\subsection{Fonctions de service et de contrainte}
\subsubsection{Fonctions de service principales}
% (qui sont la raison d’être du produit)
\subsubsection{Fonctions de service complémentaires}
% (qui améliorent, facilitent ou complètent le service rendu)
\subsubsection{Contraintes}
% (limitations à la liberté du concepteur-réalisateur)
Le framework doit être au maximum indépendant, notemment au niveau de l'acces au données.
Le client la possibilité d'utiliser le systeme de gestion de base de données relationnelle (SGBDR) de son choix.

Les machines hôtes devront disposer d'une Java Virtual Machine.
Le produit bénéficie de la portabilité de Java et sera compatible avec tous les systèmes
d'exploitation pouvant disposer d'une JVM.

\subsection{Critères d’appréciation}
%(en soulignant ceux qui sont déterminants pour l’évaluation des réponses)

\subsection{Niveaux des critères d’appréciation et ce qui les caractérise}
\subsubsection{Niveaux dont l’obtention est imposée}
\subsubsection{Niveaux souhaités mais révisables}
