\documentclass[10pt,a4paper]{article}

\usepackage[utf8]{inputenc}
\usepackage[francais]{babel}



% Title Page
\title{Projet PPO : Cahier des Charges}
\author{
	Badre Baba, Geoffroy Carrier, J-Ch Saad-Dupuy,Mickael Scheer
}


\begin{document}
\maketitle

%%NORME AFNOR
\section{Présentation Générale}
\subsection{Projet}
\subsubsection{Finalités}
\subsubsection{Espérance de retour sur investissement}

\subsection{Contexte}
\subsubsection{Situation du projet par rapport aux autres projets de l’entreprise}
\subsubsection{Études déjà effectuées}
\subsubsection{Études menées sur des sujets voisins}
\subsubsection{Suites prévues}
\subsubsection{Nature des prestations demandées}
\subsubsection{Parties concernées par le déroulement du projet et ses résultats}
% (demandeurs, utilisateurs)

\subsection{Enoncé du besoin}
%(finalités du produit pour le futur utilisateur tel que prévu par le demandeur)

\subsection{Environnement du produit recherché}
\subsubsection{Listes exhaustives des éléments} 
%(personnes, équipements, matières…) et contraintes (environnement)
\subsubsection{Caractéristiques pour chaque élément de l’environnement}

\section{Expression fonctionnelle du besoin}

\subsection{Fonctions de service et de contrainte}
\subsubsection{Fonctions de service principales}
% (qui sont la raison d’être du produit)
\subsubsection{Fonctions de service complémentaires}
% (qui améliorent, facilitent ou complètent le service rendu)
\subsubsection{Contraintes}
% (limitations à la liberté du concepteur-réalisateur)

\subsection{Critères d’appréciation}
%(en soulignant ceux qui sont déterminants pour l’évaluation des réponses)

\subsection{Niveaux des critères d’appréciation et ce qui les caractérise}
\subsubsection{Niveaux dont l’obtention est imposée}
\subsubsection{Niveaux souhaités mais révisables}

\subsection{Cadre de réponse}

\subsubsection{Pour chaque fonction}
\begin{enumerate}
\item{Solution proposée}
\item{Niveau atteint pour chaque critère d’appréciation de cette fonction et modalités de contrôle}
\item{Part du prix attribué à chaque fonction}
\end{enumerate}

\subsubsection{Pour l’ensemble du produit}
\begin{enumerate}
\item{Prix de la réalisation de la version de base}
\item{Options et variantes proposées non retenues au cahier des charges}
\item{Mesures prises pour respecter les contraintes et leurs conséquences économiques}
\item{Outils d’installation, de maintenance … à prévoir}
\item{Décomposition en modules, sous-ensembles}
\item{Prévisions de fiabilité}
\item{ Perspectives d’évolution technologique}
\end{enumerate}

\end{document}
